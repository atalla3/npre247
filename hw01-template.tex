% use the answers clause to get answers to print; otherwise leave it out.
\documentclass[12pts,answers,addpoints]{exam}
%\documentclass[12pts]{exam}
\RequirePackage{amssymb, amsfonts, amsmath, latexsym, verbatim, xspace, setspace, mathrsfs}
\usepackage{graphicx}

% By default LaTeX uses large margins.  This doesn't work well on exams; problems
% end up in the "middle" of the page, reducing the amount of space for students
% to work on them.
\usepackage[margin=1in]{geometry}
\usepackage{enumerate}
\usepackage[hidelinks]{hyperref}

% Here's where you edit the Class, Exam, Date, etc.
\newcommand{\class}{NPRE 247}
\newcommand{\term}{Fall 2018}
\newcommand{\assignment}{HW 1}
\newcommand{\duedate}{2018.08.31}
%\newcommand{\timelimit}{50 Minutes}

\newcommand{\nth}{n\ensuremath{^{\text{th}}} }
\newcommand{\ve}[1]{\ensuremath{\mathbf{#1}}}
\newcommand{\Macro}{\ensuremath{\Sigma}}
\newcommand{\vOmega}{\ensuremath{\hat{\Omega}}}

% For an exam, single spacing is most appropriate
\singlespacing
% \onehalfspacing
% \doublespacing

% For an exam, we generally want to turn off paragraph indentation
\parindent 0ex

%\unframedsolutions

\begin{document} 

% These commands set up the running header on the top of the exam pages
\pagestyle{head}
\firstpageheader{}{}{\makebox[0.5\textwidth]{\hfill Name: \underline{YOUR NAME HERE}}}
\runningheader{\class}{\assignment\ - Page \thepage\ of \numpages}{Due \duedate}
\runningheadrule

\class \hfill \term \\
\assignment \hfill Due \duedate\\
\rule[1ex]{\textwidth}{.1pt}
%\hrulefill

%%%%%%%%%%%%%%%%%%%%%%%%%%%%%%%%%%%%%%%%%%%%%%%%%%%%%%%%%%%%%%%%%%%%%%%%%%%%%%%%%%%%%
%%%%%%%%%%%%%%%%%%%%%%%%%%%%%%%%%%%%%%%%%%%%%%%%%%%%%%%%%%%%%%%%%%%%%%%%%%%%%%%%%%%%%
\begin{itemize}
        \item Show your work. 
        \item This work must be submitted online as a \textbf{single} 
                \texttt{.pdf} file through Compass2g.
        \item Work completed with LaTeX or Jupyter earns 1 extra point. Submit 
                source file (e.g. \texttt{.tex} or \texttt{.ipynb}) along with 
                the \texttt{.pdf} file.
        \item If this work is completed with the aid of a numerical program 
                (such as Python, Wolfram Alpha, or MATLAB) all scripts and data 
                must be submitted in addition to the \texttt{.pdf}.
        \item If you work with anyone else, document what you worked on together.
\end{itemize}
\rule[1ex]{\textwidth}{.1pt}

% ---------------------------------------------
\begin{questions}
        \question Myriad helpful facts can be learned from the syllabus.
        \begin{parts}
                \part[2] What are you hoping to learn in this class?
                \begin{solution}
                        YOUR SOLUTION HERE.
                \end{solution}
                \part[2] By what day and time every week must online quizzes be 
                completed?
                \begin{solution}
                        YOUR SOLUTION HERE.
                \end{solution}
                \part[2] When is homework due every week?
                \begin{solution}
                        YOUR SOLUTION HERE.
                \end{solution}

        \end{parts}

        % ---------------------------------------------
        % intro
        \question 
        \begin{parts}
                \part[2] In your opinion, what is the most compelling reason to support nuclear 
                power?
                \begin{solution}
                        YOUR SOLUTION HERE.
                \end{solution}
                \part[2] In your opinion, what is the most compelling reason to oppose nuclear 
                power?
                \begin{solution}
                        YOUR SOLUTION HERE.
                \end{solution}
        \end{parts}

        %------------------------------------
        %------------------------------------
        \question 
        Domestic electricity consumption data can be found at 
        \href{http://eia.gov}{eia.gov}. 
        \begin{parts}
                \part[5] How many gigawatts are in 1000 megawatts?
                \begin{solution}
                        YOUR SOLUTION HERE.
                \end{solution}
                \part[10] Using the data you can find with the electricity data 
                browser, report the May 2018 domestic (United States) electricity generation from all sources in gigawatt hours.
                \begin{solution}
                        YOUR SOLUTION HERE.
                \end{solution}

                \part[10] Using the data you can find with the electricity data 
                browser, report the May 2018 domestic (United States) electricity 
                generation from nuclear in gigawatt hours.
                \begin{solution}
                        YOUR SOLUTION HERE.
                \end{solution}

                \part[5] So, then, what percentage of domestic generation in May 
                2018 was nuclear? Please report as a percentage, with 
                resolution to the hundredth of a percent.
                \begin{solution}
                        YOUR SOLUTION HERE.
                \end{solution}
        \end{parts}

\question[5] International electricity consumption data can be found at \href{http://iea.org}{iea.org}.
        Using the statistics you can find on this website, 
                report the 2015 worldwide electricity generation from all 
                sources in gigawatt hours.
        \begin{solution}
                        YOUR SOLUTION HERE.
                        A math example is below for your convenience. Imagine 
                        $x=2$.
                \begin{align*}
                        y &= mx + b\\
                          &= 2x + 10\\
                          &= 4 + 10\\
                          &= \boxed{14}\\
                \end{align*}
        \end{solution}

%------------------------------------
\question[5] (Shultis \& Faw, 1.1) Both the hertz and the curie have units of 
($s^{-1}$). Explain the difference between these two units.
\begin{solution}
                        YOUR SOLUTION HERE.
\end{solution}

%------------------------------------
\question[10] (Shultis \& Faw, 1.17) How many atoms of deuterium are there in 
2kg of water?
\begin{solution}
                        YOUR SOLUTION HERE.
\end{solution}

%------------------------------------
\question (Shultis \& Faw, 1.23) A reactor is fueled with 4 kg uranium enriched to 20 atom-percent in 
$^{235}U$. The remainder of the fuel is $^{238}U$. The fuel has a mass density 
of 19.2 $\frac{g}{cm^3}$.
\begin{parts}
        \part[15] What is the mass of $^{235}U$ in the reactor?
        \begin{solution}
                        YOUR SOLUTION HERE.
        \end{solution}
        \part[15] What are the atom densities of $^{235}U$ and $^{238}U$ in the fuel?
        \begin{solution}
                        YOUR SOLUTION HERE.
        \end{solution}
\end{parts}

%------------------------------------
\question[10] (Shultis \& Faw, 1.27) How much larger in diameter is a uranium 
nucleus compare to an iron nucleus?
        \begin{solution}
                        YOUR SOLUTION HERE.
        \end{solution}
\end{questions}



%\bibliographystyle{plain}
%\bibliography{hw01}
\end{document}
